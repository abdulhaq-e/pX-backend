\documentclass[fontsize=14,headinclude=true, headsepline=true,
footsepline=true]{scrartcl}
\usepackage{scrlayer-scrpage}
\usepackage[paperwidth=210mm,paperheight=297mm,left=25mm,right=25mm,
          top=30mm,bottom=30mm,footskip=10mm,headsep=0mm]{geometry}
\usepackage{fontspec}
\usepackage{polyglossia}
\usepackage{booktabs}
\usepackage{amssymb}
\usepackage{graphicx}
\usepackage{array}
\usepackage[inline]{enumitem}
\usepackage{dashrule}
\setlist{nosep}
%\usepackage[arabic,english]{babel}
\setdefaultlanguage{english}
\setotherlanguage[numerals=maghrib]{arabic}
%\csname @Latintrue\endcsname

\setmainfont[Ligatures=TeX]{Scheherazade}%{Scheherazade}%{Amiri}
\setsansfont{Scheherazade}%Al-Kharashi 12}%Scheherazade}
\newfontfamily\arabicfont[Script=Arabic,Scale=1.2]{Scheherazade}%Adobe Arabic}

\pagestyle{scrheadings}
\rohead*{\bfseries\textarabic{جامعة طرابلس\\كلية الهندسة\\قسم هندسة الطيران}}
\lohead*{\bfseries\textarabic{مكتب الدراسة والامتحانات}}
\cfoot*{}
\lofoot*{\textarabic{تاريخ الإصدار: \today}}
\chead*{\includegraphics[scale=0.7]{/home/abdulhaq/workspace/pX/pX-backend/pX/forms/jinja2/logo.pdf}}
%\chead*{\textenglish{whatever}}

\AtBeginDocument{\begingroup}
\AtEndDocument{\endgroup}

\makeatletter

\newcolumntype{"}{@{\hskip\tabcolsep\vrule width 2pt\hskip\tabcolsep}}
\makeatother


\begin{document}
\resetdefaultlanguage{arabic}
  %{{ content }}

%\title{\textarabic{نموذج الحجز المبدئي لفصل الخريف 2015 للعام الدراسي
  %2015/2016}}
%\date{}
%\maketitle

%\begin{Arabic}
{
\bfseries
\centering{نموذج الحجز المبدئي لفصل الربيع 2016 للعام الدراسي
  2015/2016\\[0.5cm]\par
}
}
الاسم: {{ student.full_name_ar  }}

رقم القيد: {{ student.registration_number }}

اسم المشرف: {{ student.advisor }}


\textbf{ملاحظات:}
\begin{enumerate}
  \item الحد الأدنى من الوحدات 12 والحد الأقصى {{
  student.get_max_allowed_credits() }}.
  \item ضرورة مراعاة تسجيل المقررات أولاً بأول.
\item للحجز يتم اختيار المقررات من قائمة المقررات المسموح بها ومن ثم إدراجها في الجدول ويمنع
إدراج مقرر غير موجود بالقائمة.
\item حجز المواد العامة يكون عبر الطريقة المعلنة من المرحلة العامة. \textenglish{www.eng.org.ly}
  \item لحجز المقررات التالية \textenglish{ME201, ME206, ME215, ME307, EE280} يجب التسجيل في القائمة
  المبدئية عند المكان المعلن ولا يكفي حجزها في هذه القائمة.
\item يتحمل الطالب مسؤولية التعارض في الجداول الدراسية والامتحانات الدورية والنهائية.

\end{enumerate}

\textbf{المقررات المسموح بتسجيلها: (اسم المقرر | رمز المقرر | عدد الوحدات)}

\begin{table}[!ht]
  \small
  \centering
\begin{tabular}{@{}l@{}l@{}c@{}c@{}"@{}l@{}l@{}c@{}c@{}}
  
 {{loop.index}}. & \textenglish{ {{ course.name_en |replace('&','\&')|safe }} }
  & \textenglish{ {{ course.code }} } & {{course.credit}}
  {% if loop.index % 2 == 1 %}
      &
  
  {% if loop.index % 2 == 0 %}
    \\
  
  
\end{tabular}
\end{table}

\begin{table}[!ht]
   \renewcommand*{\arraystretch}{1.1}
   % \setlength{\aboverulesep}{0pt}
    %\setlength{\belowrulesep}{0pt}
    %\setlength{\extrarowheight}{.75ex}
  \centering
  \begin{tabular}{ccm{6.5cm}ccc}
    \toprule
ت & رمز المقرر & \multicolumn{1}{c}{اسم المقرر} &
 عدد الوحدات & المجموعة & موافقة الحجز
    \\
    \midrule

    
    {{ loop.index }} & & & & &\\
    
\bottomrule
  \end{tabular}
\end{table}
مجموع الوحدات المسجلة: $\ldots\ldots$

%\vspace{0.2cm}

\vspace{1cm}

\begin{minipage}[t]{0.5\textwidth}
\flushright
توقيع الطالب: $\ldots\ldots\ldots\ldots\ldots\ldots\ldots\dots$
  \end{minipage}
\hfill
\begin{minipage}[t]{0.5\textwidth}
\flushleft
التاريخ: $\ldots\ldots\ldots\ldots\ldots\ldots\ldots\dots$
\end{minipage}
\vfill

% \textbf{بيانات تعبأ من قبل المشرف:}
%
%
% ملاحظات:
%
% \fbox{
%   \begin{minipage}{\textwidth}
%     \hfill\vspace{2.5cm}
%   \end{minipage}
% }
% \vspace{0.3cm}
%

\vspace{0.3cm}

\begin{minipage}[t]{0.5\textwidth}
\flushright
توقيع المشرف: $\ldots\ldots\ldots\ldots\ldots\ldots\ldots\dots$
  \end{minipage}
\hfill
\begin{minipage}[t]{0.5\textwidth}
\flushleft
التاريخ: $\ldots\ldots\ldots\ldots\ldots\ldots\ldots\dots$
\end{minipage}


\end{document}

%%% Local Variables:
%%% mode: latex
%%% TeX-master: t
%%% End:
