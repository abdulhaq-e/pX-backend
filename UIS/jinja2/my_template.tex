\documentclass[fontsize=14,headinclude=true, headsepline=true,
footsepline=true]{scrartcl}
\usepackage{scrlayer-scrpage}
\usepackage[paperwidth=210mm,paperheight=297mm,left=25mm,right=25mm,
          top=30mm,bottom=30mm,footskip=10mm,headsep=0mm]{geometry}
\usepackage{fontspec}
\usepackage{polyglossia}
\usepackage{booktabs}
\usepackage{amssymb}
\usepackage{graphicx}
\usepackage[inline]{enumitem}
\setlist{nosep}
%\usepackage[arabic,english]{babel}
\setdefaultlanguage{english}
\setotherlanguage[numerals=maghrib]{arabic}
%\csname @Latintrue\endcsname

\setmainfont[Ligatures=TeX]{Scheherazade}%{Scheherazade}%{Amiri}
\setsansfont{Scheherazade}%Al-Kharashi 12}%Scheherazade}
\newfontfamily\arabicfont[Script=Arabic,Scale=1.2]{Scheherazade}%Adobe Arabic}

\pagestyle{scrheadings}
\rohead*{\bfseries\textarabic{جامعة طرابلس\\كلية الهندسة\\قسم هندسة الطيران}}
\lohead*{\bfseries\textarabic{مكتب الدراسة والامتحانات}}
\cfoot*{}
\lofoot*{\textarabic{تاريخ الإصدار: \today}}
\chead*{\includegraphics[scale=0.7]{/home/abdulhaq/workspace/pX/pX-backend/UIS/jinja2/logo7.pdf}}
%\chead*{\textenglish{whatever}}

\AtBeginDocument{\begingroup}
\AtEndDocument{\endgroup}

\begin{document}
\resetdefaultlanguage{arabic}
  %{{ content }}

%\title{\textarabic{نموذج الحجز المبدئي لفصل الخريف 2015 للعام الدراسي
  %2015/2016}}
%\date{}
%\maketitle

%\begin{Arabic}
{
\bfseries
\centering{نموذج الحجز المبدئي لفصل الخريف 2015 للعام الدراسي
  2015/2016\\[0.5cm]\par
}
}
الاسم: {{ student.get_full_name_ar()  }}

رقم القيد: {{ student.registration_number }}

اسم المشرف: {{ student.advisor }}


\begin{table}[!ht]
   \renewcommand*{\arraystretch}{0.8}
   % \setlength{\aboverulesep}{0pt}
    %\setlength{\belowrulesep}{0pt}
    %\setlength{\extrarowheight}{.75ex}
  \centering
  \begin{tabular}{ccccccc}
    \toprule
    ت & رقم المقرر & اسم المقرر &
 عدد الوحدات & الحجز  & المجموعة
    \\
    \midrule

    
    {{ loop.index }} & {{ course.code }} & \textenglish{ {{ course.name|replace('&','\&')
                                           }} } & {{
                                                               course.credit
                                                               }} &
                                                                    \framebox(10,10){} & \\
    
\bottomrule
  \end{tabular}
\end{table}

%\vspace{0.2cm}

ملاحظات:

\begin{enumerate}
\item الحد الأدنى من الوحدات 12 والحد الأقصى {{
      student.get_max_enrolled_credits() }}.
\item ضرورة مراعاة تسجيل المقررات أولاً بأول.
\item لحجز مقرر، يرجى وضع علامة صح في خانة ''الحجز''
\textenglish{\makebox[0pt][l]{$\square$}\raisebox{.15ex}{\hspace{0.1em}$\checkmark$}}
وتعبئة خانة المجموعة عند اللزوم.
\end{enumerate}

\vspace{0.2cm}

\begin{minipage}[t]{0.5\textwidth}
\flushright
توقيع الطالب: $\ldots\ldots\ldots\ldots\ldots\ldots\ldots\dots$
  \end{minipage}
\hfill
\begin{minipage}[t]{0.5\textwidth}
\flushleft
التاريخ: $\ldots\ldots\ldots\ldots\ldots\ldots\ldots\dots$
\end{minipage}
\vfill

\textbf{بيانات تعبأ من قبل المشرف:}


ملاحظات:

\fbox{
  \begin{minipage}{\textwidth}
    \hfill\vspace{2.5cm}
  \end{minipage}
}
\vspace{0.3cm}

مجموع الوحدات المسجلة: $\ldots\ldots$

\vspace{0.3cm}

\begin{minipage}[t]{0.5\textwidth}
\flushright
توقيع المشرف: $\ldots\ldots\ldots\ldots\ldots\ldots\ldots\dots$
  \end{minipage}
\hfill
\begin{minipage}[t]{0.5\textwidth}
\flushleft
التاريخ: $\ldots\ldots\ldots\ldots\ldots\ldots\ldots\dots$
\end{minipage}


\end{document}

%%% Local Variables:
%%% mode: latex
%%% TeX-master: t
%%% End:
